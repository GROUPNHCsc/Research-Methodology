\documentclass{article}
\usepackage{float}
\usepackage{graphicx}

\begin{document}

\begin{figure}[H]
  \includegraphics[height=3in]{/Users/Dintaine/Desktop/latex/figures/muk logo.jpg}
 \end{figure}
\centerline{COLLEGE OF COMPUTING AND INFORMATION SCIENCES\\}
\centerline{DEPARTMENT OF COMPUTER SCIENCE\\}
\centerline{COURSEWORK: RESEARCH METHODOLOGY(BIT 2207)\\}
\centerline{LECTURER: MR.ERNERST MWEBAZE\\}
\cleardoublepage
\paragraph{•}
\centerline{\begin{tabular}{|c|c|c|c|}
\hline
\textbf{No.}& \textbf{Student Name} & \textbf{RegNo} &  \textbf{StdNo}\ \\ \hline
\textit{1}&\textbf{NASSIMBWA DOREEN} & \textit{16/U/10016/EVE}& \textit{216004538} \\ \hline
\textit{2}&\textbf{KATENDE PAUL }& \textit{16/U/5599/EVE}& \textit{216007292} \\ \hline
\textit{3}&\textbf{ODEKE MOSES} & \textit{16/U/10748/PS} & \textit{216016894} \\ \hline
\textit{4}&\textbf{AINEMUKAMA DINTON HAROLD}& \textit{16/U/3020/PS}  & \textit{216007270} \\ \hline
\end{tabular}}
\begin{titlepage}
	\begin{center}
	\line(1,0){300}\\
	\huge{\bfseries Implementation of an Automated Records Management System on a Poultry Farm }\\
	[2mm]
	\line(1,0){200}\\
	[1.5cm]
	\end{center}
\end{titlepage}
\section{Introduction}
\subsection{Background of Study}
The need for, use and benefits of information for farm decision making has engaged the attention of farmers, researchers and policymakers over the years. Information is data that has been transformed into a form that is meaningful and useful for decision-making with data distinguished as raw facts, figures, objects et cetera. The ‘system’ about information relates to the connection or integration of components of collection, processing, storage, and distribution of information to support decision-making. By extension of this non-farm definition, farm information systems (FIS), then, can be appreciated as a tool to assist farms in forward planning, risk management, and by the use of information. Poultry production enterprises require good information systems to ensure success.
\subsection{Statement of the Problem}
Many poultry farmers still make use of the manual approach of keeping farm records. The consequences of this approach are it is time consuming, needed information may easily be misplaced, un-organized and inefficient. Also, needed reports concerning different aspects of the farm cannot be easily retrieved when needed. This situation makes it to monitor the state of the birds in the poultry, income and expenses and other relevant information. To overcome these problems there is need for an information system for proper management of the poultry farm.
\end{document}
