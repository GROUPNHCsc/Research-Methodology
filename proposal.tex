\documentclass{article}
\usepackage{float}
\usepackage{graphicx}

\begin{document}

\begin{figure}[h]
  \centerline{\small MAKERERE 
  \includegraphics[width=0.2\textwidth]  {mak-logo-sm.png} UNIVERSITY\\}
 \end{figure}
\centerline{COLLEGE OF COMPUTING AND INFORMATION SCIENCES\\}
\centerline{DEPARTMENT OF COMPUTER SCIENCE\\}
\centerline{COURSEWORK: RESEARCH METHODOLOGY(BIT 2207)\\}
\centerline{LECTURER: MR.ERNERST MWEBAZE\\}
\cleardoublepage
\paragraph{•}
\centerline{\begin{tabular}{|c|c|c|c|}
\hline
\textbf{No.}& \textbf{Student Name} & \textbf{RegNo} &  \textbf{StdNo}\ \\ \hline
\textit{1}&\textbf{NASSIMBWA DOREEN} & \textit{16/U/10016/EVE}& \textit{216004538} \\ \hline
\textit{2}&\textbf{KATENDE PAUL }& \textit{16/U/5599/EVE}& \textit{216007292} \\ \hline
\textit{3}&\textbf{ODEKE MOSES} & \textit{16/U/10748/PS} & \textit{216016894} \\ \hline
\textit{4}&\textbf{AINEMUKAMA DINTON HAROLD}& \textit{16/U/3020/PS}  & \textit{216007270} \\ \hline
\end{tabular}}
\begin{titlepage}
	\begin{center}
	\line(1,0){300}\\
	\huge{\bfseries Implementation of an Automated Records Management System on a Poultry Farm }\\
	[2mm]
	\line(1,0){200}\\
	[1.5cm]
	\end{center}
\end{titlepage}
\section{Introduction}
\subsection{Background of Study}
The need for, use and benefits of information for farm decision making has engaged the attention of farmers, researchers and policymakers over the years. Information is data that has been transformed into a form that is meaningful and useful for decision-making with data distinguished as raw facts, figures, objects et cetera. The ‘system’ about information relates to the connection or integration of components of collection, processing, storage, and distribution of information to support decision-making. By extension of this non-farm definition, farm information systems (FIS), then, can be appreciated as a tool to assist farms in forward planning, risk management, and by the use of information. Poultry production enterprises require good information systems to ensure success.
\subsection{Statement of the Problem}
Many poultry farmers still make use of the manual approach of keeping farm records. The consequences of this approach are it is time consuming, needed information may easily be misplaced, un-organized and inefficient. Also, needed reports concerning different aspects of the farm cannot be easily retrieved when needed. This situation makes it to monitor the state of the birds in the poultry, income and expenses and other relevant information. To overcome these problems there is need for an information system for proper management of the poultry farm.
\subsection{Objectives}
\subsubsection{General objectives}
To develop an automated poultry keeping information system.
\subsubsection{Specific Objectives}
\begin{itemize}
  \item To automate the manual means of recording poultry farm records.
  \item To develop a database application that can be used to maintain and provide information about livestock and financial information aspect of the poultry farm.
  \item To provide a system that can facilitate the update of poultry farm records.
  \item To develop a system that will aid the presentation of reports pertaining the poultry farm.
\end{itemize}
\subsection{Scope of Study}
This research work covers Design and Implementation of a poultry keeping information system. It is restricted to recording information concerning the birds reared and the financial aspect of income and expenses of the poultry farm.
\subsection{Significance of the Study}
The significance of the study is that it will provide useful information and means to enable the management of the poultry farm automate their record keeping process for better updating and presentation of reports. It will also serve as a useful reference material to other researchers that need related information.
\section{Literature review}
\subsection{Introduction}
Record keeping and management at a farm involves keeping an account of the daily
operations for reference. Torres defined record keeping as the keeping of detailed
records by a farmer of his farm’s daily operations, incomes and expenses \cite{r2}. These records are important because they guide farmer decision making. Poggio classified farm records under four basic types: resource inventories, production records, financial records and supplementary records. Of these, farmers are normally more concerned with production
records which keep an account of mortalities, production, drug administration, weight, feed
and other day to day activities \cite{r2}. All records however are important and require
attention especially in the era of commercialization of poultry farms. 
\subsection{Main Body}
Several researchers (Torres \cite{r2},Dixon and Minae \cite{r4}) have pointed out several reasons for lack of record keeping among farmers in Africa which include:
\begin{itemize}
\item The cumbersome nature of record keeping due to high illiteracy levels and low
numeracy levels of farmers and farm managers.
\item Lack of necessary skills and resources (e.g computer software).
\item Lack of a central database or reference point for farm information to provide some
form of harmonisation and coordination in data collection methodologies, indicators
and variables.
\end{itemize}
\subsection{Conclusion}
Of the 13 case studies, only three  had a computer on site. These three farms were all
using Microsoft Excel and Microsoft Word to store farm records and make performance
reports \cite{r1}. Of the 10 who had no computers on site, six said they didn’t find any value added by the presence of computers at their farms, 4 gave no reason for not having a
computer on site. All the ten respondents  with no computer agreed that they would
purchase a computer only if it added value to their businesses \cite{r1}. While all respondents were computer literate, none of the respondents was using any ICT service/application for management and decision making \cite{r1}. It was however observed that all the 13 farms had at least more than one mobile phone on site. After a further inquiry on the use of mobile phones at farms, respondents mentioned that they used mobile phones for making and receiving calls,sending messages, searching the internet for information and using social media \cite{r1}. While these finding agree with the low technology usage widely reported among farmers in East Africa \cite{r3}. The finding that no farmer was currently using any decision support ICT service or application may be an indication that current decision support services have not been widely promoted among farmers in East Africa.
\subsection{References}
\bibliographystyle{IEEEtran}
\bibliography{references1}
\end{document}
